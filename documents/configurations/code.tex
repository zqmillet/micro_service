%!TEX root = ../main.tex

\usepackage{minted}
\makeatletter
\def\inlinebox@true{true}
\define@key{inlinebox}{frame rule color} {\def\inlinebox@framerulecolor{#1}}
\define@key{inlinebox}{frame back color} {\def\inlinebox@framebackcolor{#1}}
\define@key{inlinebox}{frame text color} {\def\inlinebox@frametextcolor{#1}}
\define@key{inlinebox}{frame rule width} {\def\inlinebox@framerulewidth{#1}}
\define@key{inlinebox}{banner width}     {\def\inlinebox@bannerwidth{#1}}
\define@key{inlinebox}{show banner}[true]{\def\inlinebox@showbanner{#1}}
\define@key{inlinebox}{banner text color}{\def\inlinebox@bannertextcolor{#1}}
\define@key{inlinebox}{banner back color}{\def\inlinebox@bannerbackcolor{#1}}
\define@key{inlinebox}{banner text}      {\def\inlinebox@bannertext{#1}}
\NewDocumentCommand{\inlinebox}{O{} m}{%
  \setkeys{inlinebox}{%
    frame rule color  = black,
    frame back color  = white,
    frame text color  = black,
    frame rule width  = 0.4pt,
    banner width      = 8pt,
    show banner       = false,
    banner text color = white,
    banner back color = black,
    banner text       = BAN,
    #1
  }%
  \tcbox[%
    enhanced,
    tcbox raise base,
    nobeforeafter,
    boxrule           = \inlinebox@framerulewidth,
    top               = -1pt,
    bottom            = -1pt,
    right             = -1pt,
    arc               = 1pt,
    left              = \ifx\inlinebox@showbanner\inlinebox@true\inlinebox@bannerwidth-2pt\else-1pt\fi,
    colframe          = \inlinebox@framerulecolor,
    coltext           = \inlinebox@frametextcolor,
    colback           = \inlinebox@framebackcolor,
    before upper      = {\vphantom{蛤dg}},
    overlay           = {%
      \begin{tcbclipinterior} \ifx\inlinebox@showbanner\inlinebox@true
          \fill[\inlinebox@bannerbackcolor] (frame.south west) rectangle node[text = \inlinebox@bannertextcolor, scale = 0.4, font = \sffamily\bfseries, rotate = 90] {\inlinebox@bannertext} ([xshift = \inlinebox@bannerwidth]frame.north west);
        \fi
      \end{tcbclipinterior}%
    }%
  ]{#2}%
}
\makeatother

\let\oldinput\input
\makeatletter
\def\input@scale{1}
\define@key{input}{scale}{\renewcommand*{\input@scale}{#1}}
\RenewDocumentCommand{\input}{m o}{%
  \IfNoValueTF{#2}{%
    \oldinput{#1}%
  }{%
    \setkeys{input}{#2}%
    \scalebox{\input@scale}{\oldinput{#1}}%
  }%
}
\makeatother

\usemintedstyle[python]{default}
\newtcblisting[auto counter, number within = section]{codebox}[1][]{%
  listing only,
  enhanced,
  breakable,
  left            = 6mm,
  top             = 0mm,
  bottom          = 0mm,
  boxrule         = 1pt,
  colframe        = black!50,
  colback         = white,
  minted options  = {linenos, numbersep = 8pt, fontsize = \footnotesize, baselinestretch = 1.2, mathescape, breaklines = true, numbersep = 6.5pt},
  overlay         = {\begin{tcbclipinterior}\fill[black!25] (frame.south west) rectangle ([xshift = 6mm]frame.north west);\end{tcbclipinterior}},
  #1%
}

\newtcolorbox[use counter from = codebox, number within = section]{codefilebox}[1][]{%
  listing only,
  enhanced,
  breakable,
  left            = 6mm,
  top             = 0mm,
  bottom          = 0mm,
  boxrule         = 1pt,
  colframe        = black!50,
  colback         = white,
  minted options  = {linenos, numbersep = 8pt, fontsize = \footnotesize, baselinestretch = 1.2, mathescape, breaklines = true, numbersep = 6.5pt},
  overlay         = {\begin{tcbclipinterior}\fill[black!25] (frame.south west) rectangle ([xshift = 6mm]frame.north west);\end{tcbclipinterior}},
  #1%
}

\makeatletter
\def\codebox@caption{}
\def\codebox@label{}
\def\codebox@continuousnumber{false}
\def\codebox@true{true}
\def\codebox@language{python}
\define@key{codebox}{caption}{\renewcommand*{\codebox@caption}{#1}}
\define@key{codebox}{label}{\renewcommand*{\codebox@label}{#1}}
\define@key{codebox}{language}{\renewcommand*{\codebox@language}{#1}}
\define@key{codebox}{continuous number}[true]{\renewcommand*{\codebox@continuousnumber}{#1}}

\let\oldcodebox\codebox
\RenewDocumentCommand{\codebox}{O{}}{%
  \setkeys{codebox}{#1}%
  \oldcodebox[title = \hspace{-6mm}\small\bfseries 代码 \thetcbcounter. \codebox@caption, label = \codebox@label, minted language = \codebox@language]%
}

\let\oldendcodebox\endcodebox
\renewcommand{\endcodebox}{\oldendcodebox\noindent}

\NewDocumentCommand{\inputcode}{O{language = python} m}{%
\setkeys{codebox}{#1}%
\begin{codefilebox}[#1]
  \inputminted[linenos, numbersep = 8pt, fontsize = \footnotesize, baselinestretch = 1.2, mathescape, breaklines = true, numbersep = 6.5pt]{\codebox@language}{#2}
\end{codefilebox}\noindent%
}

\let\oldcodefilebox\codefilebox
\RenewDocumentCommand{\codefilebox}{O{}}{%
  \setkeys{codebox}{#1}%
  \oldcodefilebox[title = \hspace{-6mm}\small\bfseries 代码 \thetcbcounter. \codebox@caption, label = \codebox@label, minted language = \codebox@language]%
}
\makeatother

\usepackage{cleveref}
  \makeatletter
  \crefformat{figure}              {图 #2#1#3}
  \crefformat{section}             {第 #2#1#3 小节}
  \crefformat{table}               {表 #2#1#3}
  \crefformat{appendix}            {附录 #2#1#3}
  \crefformat{equation}            {公式 #2(#1)#3}
  \crefformat{Example}             {例 #2#1#3}
  \crefformat{tcb@cnt@codebox}     {代码 #2#1#3}
  \crefformat{tcb@cnt@codefilebox} {代码 #2#1#3}
  \makeatother

\newcommand{\qframe}[1]{\inlinebox[frame rule color = black]{\mintinline{text}{#1}}}

\newcommand{\ilfile}[1]{\inlinebox[show banner, banner back color = green!70, frame rule color = green, banner text = FILE]{\mintinline{text}{#1}}}
\newcommand{\illibrary}[1]{\inlinebox[show banner, banner back color = black!70, frame rule color = black, banner text = LIB]{\mintinline{text}{#1}}}
\newcommand{\ilclass}[1]{\inlinebox[show banner, banner back color = red!70, frame rule color = red, banner text = CLS]{\mintinline{text}{#1}}}
\newcommand{\ilfunction}[1]{\inlinebox[show banner, banner back color = blue!70, frame rule color = blue, banner text = FUNC]{\mintinline{text}{#1}}}
\newcommand{\ilcommand}[1]{\inlinebox[show banner, banner back color = black, frame rule color = black, banner text = CMD]{\mintinline{text}{#1}}}
\newcommand{\ilvariable}[1]{\inlinebox[banner back color = lightgreen, frame rule color = black]{\mintinline{text}{#1}}}
\newcommand{\ilapi}[1]{\inlinebox[show banner, banner back color = black!70, frame rule color = black, banner text = API]{\mintinline{text}{#1}}}
\newcommand{\ilmethod}[1]{\inlinebox[show banner, banner back color = black!70, frame rule color = black, banner text = MTHD]{\mintinline{text}{#1}}}
\newcommand{\iltype}[1]{\inlinebox[show banner, banner back color = lightgreen, frame rule color = green, banner text = TYPE]{\texttt{#1}}}
\newcommand{\ilnode}[1]{\inlinebox[banner back color = lightgreen, frame rule color = black]{\mintinline{text}{#1}}}


\newcommand{\ilinteger}[1]{\inlinebox[show banner, banner back color = black, frame rule color = black, banner text = INT]{\mintinline{text}{#1}}}
\newcommand{\ilstring}[1]{\inlinebox[show banner, banner back color = black, frame rule color = black, banner text = STR]{\mintinline{text}{#1}}}
\newcommand{\iljson}[1]{\inlinebox[show banner, banner back color = black, frame rule color = black, banner text = JSON]{\mintinline{text}{#1}}}
\newcommand{\illist}[1]{\inlinebox[show banner, banner back color = black, frame rule color = black, banner text = LIST]{\mintinline{text}{#1}}}

